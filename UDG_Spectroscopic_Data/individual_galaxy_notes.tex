# This file lists notes for individual UDGs 
Last Updated: Feburary 2024

If you would like more information, would like me to correct an error or would like your data included please contact me via email:
jonah.gannon@gmail.com

\section{Catalogue and Individual Galaxy Notes} \label{sec:catalogue}
When the mean $V-$band surface brightness within the half-light radius was unavailable it was calculated using the magnitude, half-light radius and equation 11 of \citet{Graham2005}.  When magnitudes/surface brightnesses were only available in $g-$band the magnitude has been transformed from $g$-band using $V=g-0.3$. Unless otherwise stated, when multiple measurements were available for the same property they were combined with weighting according to their uncertainties. Below we list individual notes for each UDG we have included in the catalogue.

\subsection{Andromeda XIX}
Andromeda XIX is a satellite of M31 and resides in the Local Group. Due to its extremely low surface brightness, it is unlikely similar analogues may be found outside of the Local Group. We note that Andromeda XIX is likely affected by tidal processes interacting with the nearby M31 \citep{Collins2020, Collins2022}. Any dynamical masses calculated with the data in the catalogue should be interpreted with caution. Due to the extremely diffuse nature of this object, the half-light radius, magnitude and surface brightness are highly uncertain. The listed stellar mass was calculated from the $V$-band magnitude in \citet{Martin2016} assuming $M_{\star}/L_{V}$ = 2. The data for this galaxy are taken from the works of \citet{Martin2016}, \citet{Collins2020} and \citet{Gannon2021}.

\subsection{Antlia II}
Antlia II is a satellite of the Milky Way and resides in the Local Group. Due to its extremely low surface brightness, it is unlikely that similar analogues will be found outside of the local group. Dynamical modelling by \citet{torrealba2019} suggests that a combination of tidal stripping and a cored dark matter profile can explain the properties of Antlia II. Due to the suggestion of tidal stripping, any dynamical mass calculated with the data should be treated with caution. A systematic search for globular clusters in this galaxy revealed no candidates, as such the globular cluster number is 0. The data for this galaxy are taken from the works of \citet{mcconnachie2012}, \citet{torrealba2019}, \citet{Ji2021} and \citet{Huang2021}.

\subsection{DF44}
DF44 is in the Coma cluster and has been one of the best-studied UDGs to date. It is one of only two UDGs that has had spatially resolved kinematic and stellar population gradients measured (the other being NGC~1052-DF2). This interest has mostly been the result of claims of a rich GC system associated with the galaxy \citet{vanDokkum2017} although there is currently some disagreement on the total GC numbers of DF44 in the literature \citep{Saifollahi2021, Saifollahi2022}. See \citet{Forbes2024} for a further discussion of these numbers. Following this work, we choose the \citet{vanDokkum2017} GC number. When quoting the $N_{\rm GC}$ from \citet{vanDokkum2017} we use the number listed in their abstract (74$\pm$18) which is slightly different to that in Table 1. We have been advised this is the correct number (P. van Dokkum, private communication). While we classify DF44 as being in the Coma cluster, its phase space positioning suggests it may just be beginning to infall as part of a small group \citep{vanDokkum2019b}. As such, some authors have classified it with low-density UDGs when considering its formation (e.g., \citealp{FerreMateu2023}).  The radial velocity was derived using $V_{\rm r} = c \times \ln{(1+z)}$ from the redshift listed in footnote 6 of \citet[$z=$0.02132]{vanDokkum2017}. The data for this galaxy are taken from the works of \citet{vanDokkum2016, vanDokkum2017, vanDokkum2019b, Gannon2021, Villaume2022, Webb2022} and \citet{Saifollahi2022}.

\subsection{DF07}
DF07 is in the Coma Cluster. The GC count is a combination of values by \citet[39.1$\pm$23.8]{Lim2018} and \citet[22$^{+5}_{-7}$]{Saifollahi2022}. The data for this galaxy are taken from the works of \citet{vanDokkum2015, Gu2018, Lim2018, Saifollahi2022} and \citet{FerreMateu2023}.

\subsection{DF17}
DF17 is in the Coma Cluster. The GC count is a combination of values by \citet[28$\pm$14]{Peng2016}, \citet[27$\pm$5]{Beasley2016b}, \citet[25$\pm$11]{vanDokkum2017} and \citet[26$^{+17}_{-7}$]{Saifollahi2022}. All values are within uncertainties of one another and are in good agreement \citep{Forbes2024}. The data for this galaxy are taken from the works of \citet{Peng2016, Beasley2016b, vanDokkum2017, Gu2018} and \citet{Saifollahi2022}.

\subsection{DF26}
DF26 is a Coma cluster galaxy. This galaxy is also known as Y093 or Yagi 093. The magnitude was calculated from $R$-band using $V = R + 0.5$ (based on Virgo dEs and Coma LSBs; \citealp{vanZee2004, Alabi2020}). Light-weighted ages and metallicities are available for this galaxy from \citet{RuizLara2018}. The data for this galaxy are taken from the works of \citet{Yagi2016, Alabi2018, Lim2018} and \citet{FerreMateu2018}. 

\subsection{DFX1}
DFX1 is in the Coma Cluster. There is currently some disagreement on the total GC numbers of DF X1 in the literature \citep{Saifollahi2021, Saifollahi2022}. See further \citet{Forbes2024} for a discussion of these numbers. Following this work, we choose the \citet{vanDokkum2017} GC number. When quoting the $N_{\rm GC}$ from \citet{vanDokkum2017} we use the number listed in their abstract which is slightly different from the number in Table 1. The radial velocity was derived using $V_{\rm R} = c \times \ln{(1+z)}$ from the redshift listed in section 2.1 of \cite{vanDokkum2017}. Note that it is likely that the stellar velocity dispersion is also affected by the barycentric correction issue described in footnote 16 of \cite{vanDokkum2019b}, however, the effect is likely small (P. van Dokkum, private communication). The data for this galaxy are taken from the works of \citet{vanDokkum2017}, \citet{Gannon2021}, \citet{Saifollahi2022} and \citet{FerreMateu2023}. 

\subsection{DGSAT-$\mathrm{I}$}
DGSAT-$\mathrm{I}$ is listed as field although we note that it is located near the Pisces--Perseus supercluster and may potentially be a `backsplash' galaxy \citep{MartinezDelgado2016, Papastergis2017, Benavides2021}. The backsplash galaxy hypothesis has been disfavoured by \citet{Janssens2022} and thus we continue to list this galaxy as a field object. Note that some of the GCs counted are more luminous than expected given a traditional GC luminosity function \citep{Janssens2022}. The data for this galaxy are taken from the works of \citet{MartinezDelgado2016, MartinNavarro2019} and \citet{Janssens2022}. 

\subsection{Hydra $\mathrm{I}$ UDG 11}
Hydra $\mathrm{I}$ UDG 11 is in the Hydra $\mathrm{I}$ cluster. The magnitude was converted to $g$ band using the listed $g-r$ colour in \citet{Iodice2020} and then transformed to $V$-band assuming $V = g-0.3$. The data for this galaxy are taken from the works of \citet{Iodice2020} and \citet{Iodice2023}.

\subsection{J130026.26+272735.2}
This UDG is in the Coma Cluster. The magnitude and surface brightness were calculated from $R$-band using $V=R+0.5$ (based on Virgo dEs and Coma LSBs; \citealp{vanZee2004, Alabi2020}). The data for this galaxy are taken from the work of \citet{Chilingarian2019}.

\subsection{NGC 1052-DF2}
We classify NGC 1052-DF2 as being in the NGC~1052 group. However, there is the possibility that it is no longer bound to the NGC~1052 group as a result of its formation mechanism (e.g., \citealp{Shen2021, vanDokkum2022}). NGC 1052-DF2 is irregular for a galaxy in having both an extremely low measured velocity dispersion \citep{vanDokkum2018, Danieli2019} and an excess of bright GCs beyond what is expected given the established GC luminosity function for normal galaxies \citep{vanDokkum2018b, Shen2021}. The addition of a weak rotational component, as allowed by the data, may help alleviate the paucity of dark matter suggested by its velocity dispersion \citep{Emsellem2019, Lewis2020, Montes2021}. Furthermore, it may currently be undergoing a tidal interaction (\citealp{Keim2021}, although see \citealp{Montes2021, Golini2024}). We note that there existed some initial controversy over the distance to NGC~1052-DF2, whereby a smaller distance can solve much of the galaxy's irregular properties (see e.g., \citealp{Trujillo2019, Monelli2019}). This controversy is now largely resolved by the deep \textit{HST} imaging of \citet{Shen2021}, with this distance being further updated in Appendix A of \citet{Shen2023}.

We adopt the recessional velocity and velocity dispersion measurements reported from the Keck/KCWI data of \citet{Danieli2019} over those reported from the VLT/MUSE data of \citet{Emsellem2019} due to Keck/KCWI having the higher instrumental resolution. When quoting GC counts, we use the number of GCs measured by \citet{Shen2021} in the traditional GC luminosity function luminosity range, which excludes the brighter GC sub-population. We adopt the stellar population properties reported from VLT/MUSE data in \citet{Fensch2019} over those reported from GTC/OSIRIS data in \citet{RuizLara2019} due to the larger field of view of VLT/MUSE being able to measure a more global value for the galaxy. Both values are in agreement. The data for this galaxy are taken from the works of \citet{vanDokkum2018, Fensch2019, Danieli2019, Shen2021} and \citet{Shen2023}.

\subsection{NGC 5846$\_$UDG1}
NGC 5846$\_$UDG1 is in the NGC~5846 group. This galaxy is also known as MATLAS-2019 \citep{Muller2020} and as NGC~5846-156 by \citet{Mahdavi2005}. Here, we have adopted the velocity dispersion and redshift from \cite{Forbes2021} rather than those measured in \cite{Muller2020} due to the higher instrumental resolution in the data used by \citet{Forbes2021}. We additionally adopt the distance/GC richness from \cite{Danieli2022} rather than that reported in \cite{Muller2021} due to the greater depth of the \textit{HST} data. The data for this galaxy are taken from the works of \citet{Forbes2019, Muller2020, Muller2021, Forbes2021, Danieli2022} and \citet{FerreMateu2023}.

\subsection{NGVSUDG-19}
NGVSUDG-19 is in the Virgo cluster. The data for this galaxy are taken from the works of \citet{Lim2020} and \citet{Toloba2023}.

\subsection{NGVSUDG-20}
NGVSUDG-20 is in the Virgo cluster. The data for this galaxy are taken from the works of \citet{Lim2020} and \citet{Toloba2023}.

\subsection{PUDG-R15}
PUDG-R15 is in the Perseus cluster. The data for this galaxy are taken from the works of \citet{Gannon2022} and \citet{FerreMateu2023}. 

\subsection{PUDG-R16}
PUDG-R16 is in the Perseus cluster. The data for this galaxy are taken from the work of \citet{Gannon2022}.

\subsection{PUDG-S74}
PUDG-S74 is in the Perseus cluster. The data for this galaxy are taken from the works of \citet{Gannon2022} and \citet{FerreMateu2023}. 

\subsection{PUDG-R84}
PUDG-R84 is in the Perseus cluster. The data for this galaxy are taken from the works of \citet{Gannon2022} and \citet{FerreMateu2023}. 

\subsection{Sagittarius dSph}
The Sagittarius dSph is a satellite of the Milky Way in the Local Group and is known to be completely tidally disrupted around the Milky Way \citep{Ibata2001}. Any mass calculated with values listed in the catalogue should be treated with extreme caution due to the lack of equilibrium in the galaxy. The data for this galaxy are taken from the works of \citet{mcconnachie2012, Karachentsev2017} and \citet{Forbes2018}.

\subsection{UDG1137+16}
UDG1137+16 is a satellite of the galaxy UGC~6594 in a group environment. It is also known as dw1137+16 by \citet{Muller2018}. It has a disturbed morphology suggestive that it is undergoing stripping \citep{Gannon2021}. Any mass calculated with the values listed in the catalogue should be treated cautiously. $M_r$ was transformed into $V$-band using stated $g-r$ colour (0.65) and $V=g-0.3$. The data for this galaxy are taken from \citet{Gannon2021} and \citet{FerreMateu2023}.

\subsection{VCC 1017}
VCC 1017 is a Virgo cluster galaxy. The data for this galaxy are taken from the works of \citet{Lim2020} and \citet{Toloba2023}.

\subsection{VCC 1052}
VCC 1052 is a Virgo cluster galaxy. It has been noted to have a peculiar morphology with the possibility of spiral arms and/or tidal features \citep{Lim2020}. The data for this galaxy are taken from the works of \citet{Lim2020} and \citet{Toloba2023}.

\subsection{VCC 1287}
VCC 1287 is a Virgo cluster galaxy. Here the GC velocity dispersion is a combination of that measured by \citet[33$^{+16}_{-10}$]{Beasley2016} and \citet[39$^{+20}_{-12}$]{Toloba2023}. Both values agree within uncertainties. The data for this galaxy are taken from the works of \citet{Beasley2016, Gannon2020, Gannon2021, Lim2020} and \citet{Toloba2023}.

\subsection{VCC 615}
VCC 615 is a Virgo cluster galaxy. The data for this galaxy are taken from the works of \citet{Lim2020} and \citet{Toloba2023}.

\subsection{VCC 811}
VCC 811 is a Virgo cluster galaxy. The data for this galaxy are taken from the works of \citet{Lim2020} and \citet{Toloba2023}.

\subsection{VLSB-B}
VLSB-B is a Virgo cluster galaxy. Note that many of the properties presented in the catalogue were updated in \cite{Toloba2023} from those listed in \cite{Toloba2018}. The data for this galaxy are taken from the works of \citet{Toloba2018, Lim2020} and \citet{Toloba2023}. 

\subsection{VLSB-D}
VLSB-D is a Virgo cluster galaxy. It has an elongated structure and velocity gradient \citep{Toloba2018} that suggests it is undergoing tidal stripping. Any dynamical mass derived with the properties listed must be treated with caution. Note that many of the properties presented in the catalogue were updated in \cite{Toloba2023} from those listed in \cite{Toloba2018}. It is worth noting that while this galaxy has an estimated GC number of 13 $\pm$~6.9, 14 GCs have been confirmed spectroscopically. The data for this galaxy are taken from the works of \citet{Toloba2018, Lim2020} and \citet{Toloba2023}. 

\subsection{WLM}
WLM is a galaxy on the outskirts of the Local Group. It is gas-rich and undergoing active star formation \citep{Leaman2009}. It also likely has a large rotation component in its dynamics \citep{Leaman2009}.
The data for this galaxy are taken from \citet{mcconnachie2012} and \citet{Forbes2018}. 

\subsection{Yagi 098}
Yagi 098 is a Coma cluster galaxy. The magnitude was calculated from $R$-band using $V = R + 0.5$ (based on Virgo dEs and Coma LSBs; \citealp{vanZee2004, Alabi2020}). The data for this galaxy are taken from the works of \citet{Yagi2016, Alabi2018} and \citet{FerreMateu2018}. 

\subsection{Yagi 275}
Yagi 275 is a Coma cluster galaxy. The magnitude was calculated from $R$-band using $V = R + 0.5$ (based on Virgo dEs and Coma LSBs; \citealp{vanZee2004, Alabi2020}). The data for this galaxy are taken from the works of \citet{Yagi2016, Alabi2018, Chilingarian2019} and \citet{FerreMateu2018}. 

\subsection{Yagi 276}
Yagi 276 is a Coma cluster galaxy. The magnitude was calculated from $R$-band using $V = R + 0.5$ (based on Virgo dEs and Coma LSBs; \citealp{vanZee2004, Alabi2020}). The data for this galaxy are taken from the works of \citet{Yagi2016, Alabi2018} and \citet{FerreMateu2018}. 

\subsection{Yagi 358}
Yagi 358 is a Coma cluster galaxy. The stellar mass was calculated from the absolute magnitude assuming $M_{\star} / L_{V} =2$. The data for this galaxy are taken from the works of \citet{vanDokkum2017}, \citet{Lim2018} and \citet{Gannon2023}.  

\subsection{Yagi 418}
Yagi 418 is a Coma cluster galaxy. The $M_V$ was calculated from $R$-band using $V = R + 0.5$ (based on Virgo dEs and Coma LSBs; \citealp{vanZee2004, Alabi2020}). Stellar population properties for this galaxy are presented in \cite{RuizLara2018} but here we prefer the \cite{FerreMateu2023} age/metallicity values due to their being mass-weighted in contrast to the \citet{RuizLara2018} light-weighted values. We note that the ages are in good agreement between the two studies, as is expected for such intermediate-to-old stellar populations. The data for this galaxy are taken from the works of \citet{Yagi2016, Alabi2018} and \citet{FerreMateu2018}. 

\subsection{Notable galaxies excluded from this catalogue}
Here we discuss several notable galaxies and studies that we exclude from this catalogue:

\begin{itemize}
    \item While we include 2 galaxies from the study of \citet{Chilingarian2019} that meet our UDG definition the remaining 6 are too bright and/or small to meet our UDG criteria. As such, they are excluded from this sample. 
    \item We exclude the galaxy PUDG-R24 from the study of \citet{Gannon2022} as it is too bright in surface brightness ($\langle \mu_{V} \rangle_{\rm e} \approx 24.35$~mag arcsec$^{-2}$) to meet our definition. In \citet{Gannon2022} the galaxy was considered a UDG as it was expected to fade into the UDG regime in the next few Gyr. 
    \item We exclude the galaxies OSG1 and OSG2 from \citet{RuizLara2018} due to their being light-weighted stellar population properties, rather than the mass-weighted properties presented herein. 
    \item We exclude the stacked UDG stellar population properties from \citet{Rong2020} as it is both 1) not the results for a single galaxy and 2) includes in the stack many objects that are too bright to meet our UDG definition. It is worth noting that many of these objects do have similar stellar surface densities to the UDGs in our catalogue, it is their predominantly younger stellar populations that result in their being too bright for the surface brightness criterion \citep{Rong2020}.
    \item We exclude the two galaxies presented in \citet{Greco2018} as: 1) the metallicities are lower limits and have not been measured and 2) the ages are not mean stellar ages but instead the age since the onset of star formation. We additionally note that the galaxy LSBG-285 presented by \citet{Greco2018} is too small to meet our UDG definition.
    \item  We exclude the UDGs presented in \citet{Trujillo2017} and \citet{Bellazzini2017} as only gas-phase metallicities and not stellar metallicities, are reported. We additionally note that both \citet{Bellazzini2017} galaxies are too bright to meet our UDG definition. 
    \item We exclude the galaxy NGC~1052-DF4 \citep{vanDokkum2019} from our catalogue as it does not meet the surface brightness cut of our UDG definition. To be specific, using the surface brightness at the effective radius and S\'ersic index for NGC~1052-DF4 reported in \citet[25.1 mag arcsec$^{-2}$ and 0.79 respectively]{Cohen2018} and equation 9 of \citet{Graham2005} we calculate an average surface brightness within the half-light radius of $\langle \mu_{V} \rangle_{\rm e} \approx 24.5$~mag arcsec$^{-2}$ which does not meet our definition. 
\end{itemize}

It is also worth noting that many UDGs have measurements such as redshift and rotation available from their associated HI disk (e.g., \citealt{Leisman2017, Spekkens2018, ManceraPina2019, ManceraPina2020, ManceraPina2022, Karunakaran2020, Gault2021, Kong2022, Obeirne2024}). Our chosen criteria for this catalogue do not include these galaxies as we wish to focus on the galaxies' stellar population properties, and not that of their HI. We do note that much may be learned by comparing the two properties (e.g., \citealp{KadoFong2022a, KadoFong2022b}) but that is beyond the scope of this work.
